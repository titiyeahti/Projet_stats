\documentclass[]{article}
\usepackage{lmodern}
\usepackage{amssymb,amsmath}
\usepackage{ifxetex,ifluatex}
\usepackage{fixltx2e} % provides \textsubscript
\ifnum 0\ifxetex 1\fi\ifluatex 1\fi=0 % if pdftex
  \usepackage[T1]{fontenc}
  \usepackage[utf8]{inputenc}
\else % if luatex or xelatex
  \ifxetex
    \usepackage{mathspec}
  \else
    \usepackage{fontspec}
  \fi
  \defaultfontfeatures{Ligatures=TeX,Scale=MatchLowercase}
\fi
% use upquote if available, for straight quotes in verbatim environments
\IfFileExists{upquote.sty}{\usepackage{upquote}}{}
% use microtype if available
\IfFileExists{microtype.sty}{%
\usepackage{microtype}
\UseMicrotypeSet[protrusion]{basicmath} % disable protrusion for tt fonts
}{}
\usepackage[margin=1in]{geometry}
\usepackage{hyperref}
\hypersetup{unicode=true,
            pdftitle={Projet - MST},
            pdfauthor={Thibaut MILHAUD \& Thomas KOWALSKI},
            pdfborder={0 0 0},
            breaklinks=true}
\urlstyle{same}  % don't use monospace font for urls
\usepackage{color}
\usepackage{fancyvrb}
\newcommand{\VerbBar}{|}
\newcommand{\VERB}{\Verb[commandchars=\\\{\}]}
\DefineVerbatimEnvironment{Highlighting}{Verbatim}{commandchars=\\\{\}}
% Add ',fontsize=\small' for more characters per line
\usepackage{framed}
\definecolor{shadecolor}{RGB}{248,248,248}
\newenvironment{Shaded}{\begin{snugshade}}{\end{snugshade}}
\newcommand{\KeywordTok}[1]{\textcolor[rgb]{0.13,0.29,0.53}{\textbf{#1}}}
\newcommand{\DataTypeTok}[1]{\textcolor[rgb]{0.13,0.29,0.53}{#1}}
\newcommand{\DecValTok}[1]{\textcolor[rgb]{0.00,0.00,0.81}{#1}}
\newcommand{\BaseNTok}[1]{\textcolor[rgb]{0.00,0.00,0.81}{#1}}
\newcommand{\FloatTok}[1]{\textcolor[rgb]{0.00,0.00,0.81}{#1}}
\newcommand{\ConstantTok}[1]{\textcolor[rgb]{0.00,0.00,0.00}{#1}}
\newcommand{\CharTok}[1]{\textcolor[rgb]{0.31,0.60,0.02}{#1}}
\newcommand{\SpecialCharTok}[1]{\textcolor[rgb]{0.00,0.00,0.00}{#1}}
\newcommand{\StringTok}[1]{\textcolor[rgb]{0.31,0.60,0.02}{#1}}
\newcommand{\VerbatimStringTok}[1]{\textcolor[rgb]{0.31,0.60,0.02}{#1}}
\newcommand{\SpecialStringTok}[1]{\textcolor[rgb]{0.31,0.60,0.02}{#1}}
\newcommand{\ImportTok}[1]{#1}
\newcommand{\CommentTok}[1]{\textcolor[rgb]{0.56,0.35,0.01}{\textit{#1}}}
\newcommand{\DocumentationTok}[1]{\textcolor[rgb]{0.56,0.35,0.01}{\textbf{\textit{#1}}}}
\newcommand{\AnnotationTok}[1]{\textcolor[rgb]{0.56,0.35,0.01}{\textbf{\textit{#1}}}}
\newcommand{\CommentVarTok}[1]{\textcolor[rgb]{0.56,0.35,0.01}{\textbf{\textit{#1}}}}
\newcommand{\OtherTok}[1]{\textcolor[rgb]{0.56,0.35,0.01}{#1}}
\newcommand{\FunctionTok}[1]{\textcolor[rgb]{0.00,0.00,0.00}{#1}}
\newcommand{\VariableTok}[1]{\textcolor[rgb]{0.00,0.00,0.00}{#1}}
\newcommand{\ControlFlowTok}[1]{\textcolor[rgb]{0.13,0.29,0.53}{\textbf{#1}}}
\newcommand{\OperatorTok}[1]{\textcolor[rgb]{0.81,0.36,0.00}{\textbf{#1}}}
\newcommand{\BuiltInTok}[1]{#1}
\newcommand{\ExtensionTok}[1]{#1}
\newcommand{\PreprocessorTok}[1]{\textcolor[rgb]{0.56,0.35,0.01}{\textit{#1}}}
\newcommand{\AttributeTok}[1]{\textcolor[rgb]{0.77,0.63,0.00}{#1}}
\newcommand{\RegionMarkerTok}[1]{#1}
\newcommand{\InformationTok}[1]{\textcolor[rgb]{0.56,0.35,0.01}{\textbf{\textit{#1}}}}
\newcommand{\WarningTok}[1]{\textcolor[rgb]{0.56,0.35,0.01}{\textbf{\textit{#1}}}}
\newcommand{\AlertTok}[1]{\textcolor[rgb]{0.94,0.16,0.16}{#1}}
\newcommand{\ErrorTok}[1]{\textcolor[rgb]{0.64,0.00,0.00}{\textbf{#1}}}
\newcommand{\NormalTok}[1]{#1}
\usepackage{graphicx,grffile}
\makeatletter
\def\maxwidth{\ifdim\Gin@nat@width>\linewidth\linewidth\else\Gin@nat@width\fi}
\def\maxheight{\ifdim\Gin@nat@height>\textheight\textheight\else\Gin@nat@height\fi}
\makeatother
% Scale images if necessary, so that they will not overflow the page
% margins by default, and it is still possible to overwrite the defaults
% using explicit options in \includegraphics[width, height, ...]{}
\setkeys{Gin}{width=\maxwidth,height=\maxheight,keepaspectratio}
\IfFileExists{parskip.sty}{%
\usepackage{parskip}
}{% else
\setlength{\parindent}{0pt}
\setlength{\parskip}{6pt plus 2pt minus 1pt}
}
\setlength{\emergencystretch}{3em}  % prevent overfull lines
\providecommand{\tightlist}{%
  \setlength{\itemsep}{0pt}\setlength{\parskip}{0pt}}
\setcounter{secnumdepth}{0}
% Redefines (sub)paragraphs to behave more like sections
\ifx\paragraph\undefined\else
\let\oldparagraph\paragraph
\renewcommand{\paragraph}[1]{\oldparagraph{#1}\mbox{}}
\fi
\ifx\subparagraph\undefined\else
\let\oldsubparagraph\subparagraph
\renewcommand{\subparagraph}[1]{\oldsubparagraph{#1}\mbox{}}
\fi

%%% Use protect on footnotes to avoid problems with footnotes in titles
\let\rmarkdownfootnote\footnote%
\def\footnote{\protect\rmarkdownfootnote}

%%% Change title format to be more compact
\usepackage{titling}

% Create subtitle command for use in maketitle
\newcommand{\subtitle}[1]{
  \posttitle{
    \begin{center}\large#1\end{center}
    }
}

\setlength{\droptitle}{-2em}
  \title{Projet - MST}
  \pretitle{\vspace{\droptitle}\centering\huge}
  \posttitle{\par}
  \author{Thibaut MILHAUD \& Thomas KOWALSKI}
  \preauthor{\centering\large\emph}
  \postauthor{\par}
  \predate{\centering\large\emph}
  \postdate{\par}
  \date{6 mai 2018}


\begin{document}
\maketitle

\subsection{Statistiques descriptives}\label{statistiques-descriptives}

\subsubsection{Comparaison
hommes/femmes}\label{comparaison-hommesfemmes}

\begin{Shaded}
\begin{Highlighting}[]
\NormalTok{data <-}\StringTok{ }\KeywordTok{read.csv}\NormalTok{(}\DataTypeTok{file =} \StringTok{"DB_binome_2.csv"}\NormalTok{);}
\NormalTok{n <-}\StringTok{ }\KeywordTok{nrow}\NormalTok{(data);}
\NormalTok{mandata <-}\StringTok{ }\KeywordTok{c}\NormalTok{();}
\NormalTok{womamdata <-}\StringTok{ }\KeywordTok{c}\NormalTok{();}
\ControlFlowTok{for}\NormalTok{ (i }\ControlFlowTok{in} \DecValTok{1}\OperatorTok{:}\NormalTok{n)}
\NormalTok{\{}
  \ControlFlowTok{if}\NormalTok{(data[i, }\StringTok{'Sexe'}\NormalTok{] }\OperatorTok{==}\StringTok{ }\DecValTok{0}\NormalTok{)}
\NormalTok{  \{}
\NormalTok{    mandata <-}\StringTok{ }\KeywordTok{c}\NormalTok{(mandata, data[i, }\StringTok{'Peche'}\NormalTok{])}
\NormalTok{  \}}
  \ControlFlowTok{else}
\NormalTok{  \{}
\NormalTok{    womamdata <-}\StringTok{ }\KeywordTok{c}\NormalTok{(womamdata, data[i, }\StringTok{'Peche'}\NormalTok{])}
\NormalTok{  \}}
\NormalTok{\}}

\KeywordTok{boxplot}\NormalTok{(mandata, womamdata)}
\end{Highlighting}
\end{Shaded}

\includegraphics{livrable_files/figure-latex/sexism-1.pdf}

\subsubsection{Distribution de la pêche en fonction de la tranche
d'âge}\label{distribution-de-la-peche-en-fonction-de-la-tranche-dage}

\begin{Shaded}
\begin{Highlighting}[]
\NormalTok{tranches =}\StringTok{ }\KeywordTok{c}\NormalTok{(}\DecValTok{0}\NormalTok{, }\DecValTok{0}\NormalTok{, }\DecValTok{0}\NormalTok{)}
\ControlFlowTok{for}\NormalTok{ (i }\ControlFlowTok{in} \KeywordTok{seq}\NormalTok{(}\DecValTok{1}\NormalTok{, n)) \{}
\NormalTok{    tranche =}\StringTok{ }\NormalTok{data[i, }\StringTok{"Age"}\NormalTok{]}
\NormalTok{    tranches[tranche }\OperatorTok{-}\StringTok{ }\DecValTok{1}\NormalTok{] =}\StringTok{ }\NormalTok{tranches[tranche }\OperatorTok{-}\StringTok{ }\DecValTok{1}\NormalTok{] }\OperatorTok{+}\StringTok{ }\NormalTok{data[i, }\StringTok{"Peche"}\NormalTok{]}
\NormalTok{\}}
\KeywordTok{barplot}\NormalTok{(tranches)}
\end{Highlighting}
\end{Shaded}

\includegraphics{livrable_files/figure-latex/unnamed-chunk-1-1.pdf}

\subsubsection{Intensité du vent}\label{intensite-du-vent}

\begin{Shaded}
\begin{Highlighting}[]
\KeywordTok{hist}\NormalTok{(data[,}\StringTok{'Noeuds'}\NormalTok{])}
\end{Highlighting}
\end{Shaded}

\includegraphics{livrable_files/figure-latex/unnamed-chunk-2-1.pdf} On
dirait une loi de poisson.

\subsubsection{Quantité de pêche}\label{quantite-de-peche}

\begin{Shaded}
\begin{Highlighting}[]
\KeywordTok{hist}\NormalTok{(data[,}\StringTok{'Peche'}\NormalTok{])}
\end{Highlighting}
\end{Shaded}

\includegraphics{livrable_files/figure-latex/unnamed-chunk-3-1.pdf} On
dirait une loi Normale.

\subsection{Statistiques
Inférentielles}\label{statistiques-inferentielles}

\subsubsection{Le vent}\label{le-vent}

\begin{Shaded}
\begin{Highlighting}[]
\CommentTok{#On regarde la cohérence par rapport à la loi de poisson}
\NormalTok{a <-}\StringTok{ }\KeywordTok{seq}\NormalTok{(}\DecValTok{0}\NormalTok{,}\DecValTok{8}\NormalTok{,}\DecValTok{1}\NormalTok{)}
\NormalTok{lambda =}\StringTok{ }\KeywordTok{mean}\NormalTok{(data[,}\StringTok{'Noeuds'}\NormalTok{])}

\KeywordTok{hist}\NormalTok{(data[,}\StringTok{'Noeuds'}\NormalTok{],}\DataTypeTok{freq=}\OtherTok{FALSE}\NormalTok{,}\DataTypeTok{breaks =} \KeywordTok{seq}\NormalTok{(}\DecValTok{0}\NormalTok{,}\DecValTok{8}\NormalTok{,}\DecValTok{1}\NormalTok{))}
\KeywordTok{par}\NormalTok{(}\DataTypeTok{new=}\OtherTok{TRUE}\NormalTok{)}
\KeywordTok{plot}\NormalTok{(a,}\KeywordTok{dpois}\NormalTok{(a,lambda),}\StringTok{"l"}\NormalTok{,}\DataTypeTok{col=}\StringTok{"red"}\NormalTok{)}
\end{Highlighting}
\end{Shaded}

\includegraphics{livrable_files/figure-latex/unnamed-chunk-4-1.pdf}

\paragraph{Vraisemblance}\label{vraisemblance}

Soit \(X\) un echantillon de taille \(n\) suivant une loi de poisson de
paramêtre \(\lambda\), alors sa vraisemblance vaut : \[
L_\lambda(X) = \prod_{i = 1}^n \exp(-\lambda)\frac{\lambda^{x_i}}{x_i!} = \exp(-n\lambda)\frac{\lambda^{\sum x_i}}{\prod x_i!}
\] d'où, \[
\mathcal L_\lambda(X) = \log(L_\lambda(X)) = -n\lambda + \log \lambda \sum x_i - \sum \log x_i!
\] Ainsi en dérivant \(\mathcal L_\lambda(X)\) par rapport à
\(\lambda\), on obtient : \[
{\partial \mathcal L_\lambda(X) \over \partial \lambda } = -n + \frac{\sum x_i}{\lambda}
\] et \[
{\partial^2 \mathcal L_\lambda(X) \over \partial \lambda^2} = -\frac{\sum x_i}{\lambda^2} \leq 0
\] La log-vraisemblance est donc concave ce qui signifie que les points
où la dérivée s'annule sont des maximums globaux. Ainsi, \[
\lambda_\max = {\sum^n_{i=1} x_i \over n}
\]

\begin{Shaded}
\begin{Highlighting}[]
\NormalTok{log_L =}\StringTok{ }\ControlFlowTok{function}\NormalTok{(x, l) \{}
\NormalTok{  s =}\StringTok{ }\DecValTok{0}
  \ControlFlowTok{for}\NormalTok{(i }\ControlFlowTok{in} \KeywordTok{seq}\NormalTok{(}\DecValTok{1}\NormalTok{, }\KeywordTok{length}\NormalTok{(x)))}
\NormalTok{  \{}
\NormalTok{    s =}\StringTok{ }\NormalTok{s }\OperatorTok{+}\StringTok{ }\KeywordTok{log}\NormalTok{(}\KeywordTok{factorial}\NormalTok{(x[i]))}
\NormalTok{  \}}
  \KeywordTok{return}\NormalTok{(}\OperatorTok{-}\DecValTok{1} \OperatorTok{*}\StringTok{ }\KeywordTok{length}\NormalTok{(x) }\OperatorTok{*}\StringTok{ }\NormalTok{l }\OperatorTok{+}\StringTok{ }\KeywordTok{log}\NormalTok{(l) }\OperatorTok{*}\StringTok{ }\KeywordTok{sum}\NormalTok{(x) }\OperatorTok{-}\StringTok{ }\NormalTok{s)}
\NormalTok{\}}

\NormalTok{x =}\StringTok{ }\KeywordTok{seq}\NormalTok{(}\DecValTok{0}\NormalTok{, }\DecValTok{10}\NormalTok{, }\FloatTok{0.1}\NormalTok{)}
\KeywordTok{plot}\NormalTok{(}\DataTypeTok{x =}\NormalTok{ x, }
     \DataTypeTok{y =} \KeywordTok{log_L}\NormalTok{(data[, }\StringTok{"Noeuds"}\NormalTok{], x),}
     \DataTypeTok{main =} \StringTok{"Vraisemblance en fonction de lambda"}\NormalTok{,}
     \DataTypeTok{xlab =} \StringTok{"lambda"}\NormalTok{,}
     \DataTypeTok{ylab =} \StringTok{"L_lambda(X)"}\NormalTok{,}
     \DataTypeTok{type =} \StringTok{"l"}\NormalTok{)}
\end{Highlighting}
\end{Shaded}

\includegraphics{livrable_files/figure-latex/unnamed-chunk-5-1.pdf}

\paragraph{Lambda}\label{lambda}

Par en appliquant la formule trouvée ci-dessu, on obtient :

\begin{Shaded}
\begin{Highlighting}[]
\NormalTok{lambda =}\StringTok{ }\KeywordTok{mean}\NormalTok{(data[,}\StringTok{'Noeuds'}\NormalTok{])}
\KeywordTok{print}\NormalTok{(lambda)}
\end{Highlighting}
\end{Shaded}

\begin{verbatim}
## [1] 1.724
\end{verbatim}

\begin{Shaded}
\begin{Highlighting}[]
\KeywordTok{par}\NormalTok{(}\DataTypeTok{new =} \OtherTok{FALSE}\NormalTok{)}
\KeywordTok{plot}\NormalTok{(x, x }\OperatorTok{*}\StringTok{ }\NormalTok{lambda }\OperatorTok{/}\StringTok{ }\NormalTok{x, }\DataTypeTok{type =} \StringTok{"l"}\NormalTok{)}
\end{Highlighting}
\end{Shaded}

\includegraphics{livrable_files/figure-latex/unnamed-chunk-6-1.pdf}

\subsubsection{Les poissons}\label{les-poissons}

Soit \(X\) un echantillon de taille \(n\) suivant une loi normale
d'écart-type \(\sigma\) et d'espérance \(\mu\) :
\(X\sim\mathcal{N}(\mu, \sigma^2)\), alors sa vraisemblance vaut : \[
L_\sigma(X) = \prod_{i = 1}^n \frac{1}{\sqrt{2\pi}\sigma} e^{-\frac{(x_i-\mu)^2}{2\sigma^2}} \\
            = \left( \frac{1}{\sqrt{2\pi}\sigma} \right) ^n \exp\left({-\sum_{i=1}^n \frac{x_i-\mu}{2\sigma^2}}\right)
\] d'où \[
\mathcal L_\sigma(X) = -n (\log\sqrt{2\pi} + \log\sigma) + \frac{1}{2\sigma^2}\sum_{i = 1}^n (x_i-\mu)^2
\] Ainsi en dérivant \(\mathcal L_\lambda(X)\) par rapport à
\(\lambda\), on obtient : \[
{\partial \mathcal L_\lambda(X) \over \partial \lambda } = -n + \frac{\sum x_i}{\lambda}
\] et \[
{\partial^2 \mathcal L_\lambda(X) \over \partial \lambda^2} = -\frac{\sum x_i}{\lambda^2} \leq 0
\] La log-vraisemblance est donc concave ce qui signifie que les points
où la dérivée s'annule sont des maximums globaux. Ainsi, \[
\lambda_\max = {\sum^n_{i=1} x_i \over n}
\]


\end{document}
